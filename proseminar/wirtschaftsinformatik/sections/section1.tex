\documentclass[../main.tex]{subfiles}
\begin{document}

\subsection{Motivation}
Im Zuge der Globalisierung sind viele Unternehmen vom hohen Wettbewerbsdruck betroffen \autocite[vgl.][]{alt2013handbuch}. Um ihren Ertrag langfristig zu sichern und innovativ gegenüber der Konkurrenz zu bleiben, bieten viele Industrieunternehmen ,,Value Added Services``, um die steigende Dienstleistungsnachfrage ihrer Kunden zu befriedigen \autocite[vgl.][S. 4]{Meffert2015}. Das Erbringen dieser Dienstleistungen erfordert Geschäftsprozesse die zwischenbetrieblich ausgerichtet sind \autocite[vgl.][S.19]{fleisch2001netzwerkunternehmen}. Im Kontext einer vernetzten Welt \autocite[vgl.][]{bmwi2013} ist es von besonderer Bedeutung, dass zwischenbetriebliche Daten- und Funktionsintegration auch erfolgen kann, wenn Daten über das World Wide Web ausgetauscht werden. 

\subsection{Methodik}
Das Ziel der vorliegenden Arbeit ist ein Literatur-Review durchzuführen, um die relevante Literatur über das Resource Description Framework auszuwählen, zu analysieren und zusammenzufassen im Hinblick auf den möglichen Einsatz von RDF für semantische Annotation von Daten innerhalb kleiner und mittelständischer Unternehmen. Das Review wird anhand der Methodik von \cite{fettke2006state} durchgeführt. Eine strukturierte Literatursuche wurde mithilfe der Richtlinien von \cite[S. XVI]{webster2002analyzing} umgesetzt, um eine Liste von relevanten Literatur zu erstellen. 
\begin{enumerate}
	\item Im ersten Schritt wurden \emph{EBSCOHost}, \emph{Google Scholar} und \emph{W3C Recommendations} verwendet, um eine Literaturliste zu erstellen.
	\item Danach wurden alle Quellen der Literaturliste untersucht, um weitere relevante Literatur zu finden (sogenannte \hyphenquote{german}{Go Backward} Ansatz).
	\item Schließlich wurden Google-Scholar und Semantic-Scholar benutzt, um weitere Publikationen zu identifizieren, die auf der Literaturliste von Schritt eins und zwei verweisen (sogenannte \hyphenquote{german}{Go Forward} Ansatz).
\end{enumerate}
Die folgenden Suchschlüsseln wurden während der Literatursuche benutzt:
\begin{itemize}
	\item Primärschlüssel: \texttt{Resource Description Framework, RDF, JSON-LD, RDFa}
	\item Sekundärschlüssel: \texttt{semantic annotation, semantic web, linked data}
\end{itemize}
Zum Schluss wurde die Liste der relevanten Literatur mit Blick auf der Zielstellung verarbeitet und nach wichtige Konzepte systematisiert \autocite[vgl.][S. XVI]{webster2002analyzing}.

\subsection{Aufbau}
Nach der Einleitung wird die Entwicklung von RDF im zweiten Abschnitt kurz beschrieben, und die benötigte RDF-Grundlagen und zusammenhängende Begrifflichkeiten werden herausgearbeitet. Die Syntax von RDF wird auch anhand einiger Beispiele demonstriert. Das Konzept des Semantic Webs und der Linked Data wird im dritten Abschnitt vorgestellt. Das Resource Description Framework in Attributes (RDFa) wird als Serialisierungssyntax ausgewählt um die Konzepte der semantischen Annotation im Web of Data exemplarisch zu zeigen. Im letzten Abschnitt werden die Kernkonzepte der Arbeit zusammengefasst und es wird beschrieben, welche Weiterentwicklungen künftig von Interesse sein könnten.
\end{document}
